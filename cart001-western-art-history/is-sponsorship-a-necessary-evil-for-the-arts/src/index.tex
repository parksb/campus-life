\maketitle

\begin{abstract}
  본고에서는 예술에 후원이 필요악이라는 대중적 인식을 논하고자, 서구 미술후원의 유형과 역사를 살펴본다. 고대부터 오늘날까지 미술후원은 정치적, 경제적 목적을 동기로 행해지곤 했다. 미술은 정치적 프로파간다로서 매우 효과적인 수단이기도 하여 권력자는 특정 미술가를 후원함으로써 통치의 정당성과 권력의 정통성을 선전했다. 상업이 성장한 르네상스에는 미술품이 교환가치를 지니게 되었고, 경제적 이익이 미술후원과 제작의 큰 동기가 되었다. 이후 시장경제가 자리 잡으면서부터는 미술품이 후원에서 벗어나 시장에서 거래되기 시작했다. 오늘날에는 사익으로부터 독립하여 공공의 문화예술적 가치를 실현하기 위한 공공후원이 이뤄지고 있다. 미술후원이 필요악 이상의 역할을 하기 위해서는 재단을 통한 공공후원이 지속 및 확대되어야 할 것으로 보인다.
\end{abstract}

\tableofcontents

\newpage

\section{서론}

지난 수 세기에 걸쳐 예술과 후원은 밀접한 관계를 맺으며 창작자와 후원자 사이에 공생적 생태계를 형성해 왔다. 고대부터 후원자는 모종의 목적으로 창작자를 후원했고, 창작자는 그런 후원을 바탕으로 미술품을 제작할 수 있었다. 그러나 한편으로 후원은 창작자의 자율성을 가로막고, 문화예술적 가치보다는 후원자의 사익을 추구하도록 만들어 미술의 순수성을 훼손하기도 한 것이 사실이다. 후원의 이러한 측면으로 인해 일각에서는 예술에 있어 후원을 필요악으로 여기기도 한다. 본고에서는 후원이 미술에 현실적으로는 필요하지만, 본질적으로 예술적 순수성을 위협하는 필요악이라는 대중적 인식에 관하여 논하고자 한다.

이러한 맥락에서 후원과 미술이 어떤 관계를 맺어왔는지 살펴보기 위해 미술후원의 다양한 유형을 중심으로 시대별, 후원 주체별, 목적별 후원 형태를 분석하겠다. 먼저 정치적 이익을 얻기 위한 미술후원의 역사와 사례를 돌아본 뒤, 경제적 이익을 실현하기 위한 미술후원, 그리고 공적인 목적을 위해 공공자금을 투입하는 방식의 미술후원을 차례로 살펴볼 것이다. 마지막으로 예술재단과 기업의 협력을 바탕으로 한 공공후원이 예술적 가치를 추구함으로써, 후원이 필요악 이상의 역할을 할 수 있음을 제시하며 마무리하겠다.

\section{본론}

오늘날 일반적인 후원의 정의는 어떠한 대가 없이 자선적 차원에서 자금을 제공하는 것을 의미하지만, 전근대 미술후원의 형태를 살펴보면 오늘날 기준으로는 사실상 아웃소싱처럼 보이기도 한다. 이러한 전근대 미술후원을 논의의 범위로 포함하기 위해 후원의 의미를 확장하여 미술품 제작을 위해 창작자에게 인적, 물적 자원을 제공하는 총체적 지원 행위를 미술후원으로 정의하고자 한다. 따라서 앞으로 논할 미술후원은 시대적 배경에 따라 국책사업의 형태로 나타나기도, 조직의 외주하청 형태로 나타나기도, 또는 개인의 제작 의뢰로 나타나기도 한다는 점을 미리 전제한다.

\subsection{미술후원의 정치성}

미술은 정치적 프로파간다로서 매우 효과적인 수단이기도 하다. 미술가는 추상적인 이데올로기를 시각적인 메타포로 표현하여 감상자에게 직접적으로 메시지를 전달할 수도 있고, 그림의 심미성 뒤에 이데올로기적 요소를 완곡하게 표현해 간접적으로 메시지를 전달할 수도 있다. 고대부터 권력자는 정치적 목적을 달성하고자 미술의 프로파간다적 속성을 이용했고, 이를 위해 미술가를 정치적, 경제적으로 후원해 왔다.

정치가 발달했던 고대 로마의 미술품에서는 프로파간다적 성격을 어렵지 않게 확인할 수 있다. <프리마 포르타의 아우구스투스>(14-29)의 경우 노골적으로 아우구스투스를 신격화하며 그를 승리자이자 정복자로 돋보이게 한다.\footnote{김상엽, \snm{프리마 포르타의 아우구스투스 조각상과 프로파간다}, \bnm{서양고대사연구} 52호, 한국서양고대역사문화학회, 2018, 258쪽.} 로마에서는 건축물도 다분히 정치적으로 설계되었다. 석조 조각품을 제작하는 것과 달리, 막대한 인적, 물적 자원을 필요로 하는 대규모 토건사업에는 정치적 결정이 선행되어야 했기 때문에 이는 필연적이었을 것이다. 가령 콜로세움에는 도리아 양식과 이오니아 양식, 코린트양식이 모두 사용되었는데, 콜로세움이 이렇게 화려하게 설계된 데에는 정치적 배경이 있었다. 심지어 콜로세움은 그 자체로 정치적 기능을 수행하기 위해 건설되었다. 콜로세움은 대중으로 하여금 정치에 관심을 두지 않도록 만드는 우민화의 도구이자, 검투사로 하여금 애국심과 공포심을 갖게 만드는 통제의 도구였다. 또한 중세에는 신앙심을 고취함으로써 종교 권력을 강화하거나, 왕권을 정당화하기 위한 목적으로 미술후원이 이뤄졌다. <유스티아누스 황제와 막시미아누스 주교의 입장>(547)은 황제를 그리스도처럼 표현했다. 생 라자르 성당 입구의 팀파눔 <최후의 심판>(Ghiselbertus, 1130)은 예수 그리스도와 심판받는 자들의 모습을 묘사해 성당을 방문하는 대중에게 가톨릭 질서를 충실히 따를 것을 요구한다.

르네상스 시기 사회 주도 계급으로 부상한 부르주아 계급은 그들의 부족한 정치적 기반을 문화예술로 채우고자 했다. 이들은 미술가에게 자신의 초상화를 의뢰하거나, 미술품을 수집하여 정치적 정통성을 인정받고자 했다. 이탈리아 메디치 가문은 대대로 미술가들에게 막대한 후원을 해 오늘날에는 르네상스 미술을 선도했다는 평가를 받는다. 이러한 메디치 가문의 미술후원에는 정치적 목적이 있었다. 메디치 가문은 금융업으로 막대한 부를 쌓았으나, 당시 대중은 고리대금업을 부도덕한 것으로 생각했다. 메디치 가문은 이러한 인식을 희석하기 위한 방도로 미술후원을 택했다. 특히 로렌초 데 메디치(1449-1492)의 차남이었던 교황 레오 10세(1475-1521)는 노골적으로 자신의 도덕성을 선전하고 정치적 권위를 강화하기 위해 미술을 이용했다.\footnote{이은기, \snm{메디치家의 미술후원과 정치적인 목적}, \bnm{서양미술사학회논문집} 6집, 서양미술사학회, 1994, 21쪽.} 대표적으로 레오 10세가 후원한 <샤를마뉴의 대관식>(Raphael, 1516-17)은 레오 3세가 샤를마뉴에게 로마 제국 황제의 관을 수여하는 장면을 묘사한다. 그런데 레오 10세는 레오 3세의 얼굴에 자신의 얼굴을 그릴 것을 요구하여 마치 자신이 레오 3세를 계승하는 것처럼 느껴지도록 의도했다. 800년에 진행된 이 대관식은 레오 3세가 서로마 제국의 부활을 선언하는 것이었고, 레오 10세는 그런 레오 3세의 정치적 이미지를 취하고자 한 것이다.

현대에는 국가가 미술후원의 주체로 나서며 그전과 비교할 수 없을 정도로 방대한 프로파간다 미술품이 생산되었다. 제1차 세계대전 당시 미 육군에서 제작된 포스터 <이 미치광이 짐승을 파괴하라>(Harry R. Hopps, 1917)는 고릴라로 묘사된 독일군이 백인 여성을 움켜쥔 모습을 표현해 참전의 정당성을 강조하며 육군 병사를 모집했다. 1933년 나치 독일은 정부 부처로 대중계몽선전국가부를 설치하여 시각예술에 방대한 투자를 집행했다. 전체주의 국가는 정치적 선전을 위해 자원을 `투입'하는 방식의 후원뿐 아니라, 지배 이데올로기에 반하는 미술가로부터 자원을 `회수'하고, 특정 미술 양식을 적극적으로 억압하는 부정적 후원(negative patronage)\footnote{윤난지, \snm{20세기 미술과 후원: 미국 모더니즘 정착에 있어서 구겐하임 재단의 역할을 중심으로}, \bnm{서양미술사학회논문집} 6호, 서양미술사학회, 1994, 58쪽.}을 통해 체제를 수호하기도 했다.

\subsection{미술후원의 경제성}

미술품은 미술가의 메시지, 또는 앞서 짚은 것처럼 후원자의 메시지를 전달하는 역할을 하기도 하지만, 동시에 경제적인 가치를 내재하고 있어 거래나 투자의 대상이 되기도 한다. 상업은 커녕 화폐가 등장하기 전부터 인간은 시각적으로 아름다운 물건에 높은 가치를 매겼으니 한편으로는 자연스러운 현상일 것이다. 자본주의 태동기부터 미술품은 경제적 가치를 획득하여 상품화되었고, 이후에는 미술후원 자체가 경제적 목적으로 이뤄지기 시작했다.

동인도회사를 통해 세계적인 무역국이 되어 이른바 황금시대를 맞이한 17세기 네덜란드 공화국에서는 중상주의가 확산되며 경제적 부를 축적한 부르주아 계급이 출현했다. 네덜란드의 부흥에는 종교개혁으로 주변 가톨릭 국가에서 탄압받은 개신교 상인과 예술가들이 네덜란드로 모여든 배경도 있다. 미술품의 주요 수요자였던 부르주아 계급은 미술가에게 금전적 대가를 지불하고 미술품의 생산을 의뢰하는 방식으로 미술가를 후원했다. 자본이 미술의 강력한 동력원으로 자리 잡음으로써 미술품이 높은 경제적 가치를 획득한 것이다. 당대 네덜란드의 미술가들은 대체로 부르주아 계급의 의뢰인으로부터 주문을 받고 그림을 그리곤 했다. 미술에 대한 부르주아 계급의 관심에는 앞서 살펴봤듯 정치적인 배경이 강했다.

상업의 발달은 후원자와 미술가 사이의 폐쇄적 후원 관계를 넘어 미술품을 개방된 시장으로 끌어들였고, 미술품은 시장에서 상품으로서 거래되기 시작했다.\footnote{조명계, \snm{네덜란드 미술시장 생성과정 연구}, \bnm{문화예술경영학연구} 4권 2호, 한국문화예술경영학회, 2011, 88-89쪽.} 그러나 네덜란드의 미술시장은 17세기 말 경제 침체로 인해 오래 지속되지 못했다. 따라서 미술품이 본격적으로 근대적 시장경제에 편입된 시기는 18세기로 보는 것이 일반적이다. 시장경제 체제에서 미술가들은 더 이상 후원자에게 의존하지 않고 미술품을 생산할 수 있었다. 그럼에도 이것이 예술가의 독창적인 예술세계를 구현할 수 있음을 의미하지는 않았다. 후원과 달리 수요공급의 사전적 조율이 불가능한 시장은 공급과잉을 일으켜 미술가의 생존을 위협했고, 시장에서 `팔리는' 미술품을 생산하기 위해서는 상업성을 위해 예술성을 희생해야만 했기 때문이다.\footnote{이재희, \snm{17세기 네덜란드 미술시장}, \bnm{사회경제평론} 21권, 한국사회경제학회, 2003, 266쪽.}

자본주의가 고도화되며 미술품은 투자 상품이자 자산이 되었다. 18세기에는 화상이 미술품 수집가와 수요자 사이에서 중개인 역할을 하는 사업가로 자리 잡았다.\footnote{윤자정, \snm{서구 미술시장의 양상: 역사적 고찰}, \bnm{현대미술학논문집} 16권 2호, 현대미술학회, 2012, 163쪽.} 저렴하게 매수한 미술품이 시간이 지나 재평가를 받고 가격이 오르면 차익을 실현해 경제적 이익을 얻을 수 있었고, 개인적인 소장품이었던 미술품이 경매를 통해 대중에게 공개되며 미술시장이 더욱 확장되는 순환구조를 형성했다. 20세기 기업들은 어떠한 대가도 바라지 않고 미술가들을 후원하는 것처럼 보이기 시작했다. 하지만 메세나(Mécénat)로 통칭하는 기업의 문화예술 지원에는 자선적 목적보다는 상업적 홍보 효과를 통해 경제적 이익을 얻고자 하는 목적이 더 큰 것이 사실이다. 신자유주의적 질서 속에서 메세나가 사회 공헌이 아닌, 기업의 사익 추구 욕구를 실현하기 위한 통로로 활용되고 있다는 지적은 꾸준히 제기되고 있다.

\subsection{미술후원의 공공성}

오늘날에는 정치적, 경제적 의도로부터 독립되어 예술문화의 진흥에 기여하는 공공후원도 상당히 이뤄지고 있다. 보통 국가나 재단의 공적 자금이 투입되는 공공후원도 넓게 보면 정치적, 경제적 동기가 저변에 깔려있지만, 특정 개인이나 조직의 결정으로 이뤄지는 후원보다는 민주적인 거버넌스에 의해 결정, 집행되는 후원이 공익을 추구하기에 더욱 적합할 것이다.

후원이 예술에 필요악이라는 대중적 인식을 구체화해 보면, 예술활동에 인적, 물적 자원이 필요하기 때문에 후원이 있어야 한다는 점에는 대체로 동의하지만, 예술을 이용한 후원자의 사익을 추구가 부정적인 결과를 초래한다는 점에서 가능하다면 후원이 없는 것이 더 나을 것이라는 의견에 가깝다. 하지만 예술성을 향한 미술가의 `순수한' 욕망이 아닌 정치적, 경제적 목적으로 제작된 미술품에 예술성이 없다고 할 수는 없다. 르네상스의 많은 미술품이 부르주아 계급의 의뢰를 받고 금전적인 목적으로 제작되었다. 한편으로 후원은 당대의 예술적 가치를 대변한다. 결과적으로 어떤 작품, 어떤 양식, 어떤 작가에게 후원이 이뤄지는가가 그 사회가 추구하는 예술성의 방향이라고 말할 수 있을 것이다.

미술과 후원을 분리해서 생각하기 쉽지 않을 정도로 미술후원은 그 자체로 기능을 한다. 하지만 역사적으로 후원자들이 정치적, 경제적 이익을 추구하는 과정에서 예술가의 자율성을 침해했음을 부정할 수는 없다. 앞서 살펴본 레오 10세의 요구나 전체주의 국가의 탄압은 직접적인 침해 사례라고 할만하다. 르네상스의 의뢰인들도 마찬가지였다. 가령 <야경>(Rembrandt, 1642)은 예술적으로 뛰어난 걸작으로 평가받지만, 어떤 인물은 조명을 받아 강조되고 어떤 인물은 어두운 배경처럼 그려져 당시 의뢰인들의 비난을 피할 수 없었다. 결국 주류 미술에 후원이 집중됨으로 인해 비주류 미술은 평가는 물론, 제작의 가능성조차 봉쇄되고 마는 치명적인 한계가 있는 것이다. 이러한 한계는 비단 후원뿐 아니라 시장에도 존재한다. 따라서 문화의 공공성과 다양성을 확보하면서 예술가의 자율성을 보장하기 위해서는 기업과 같은 사익조직으로부터 독립된 예술재단이 후원의 주체가 되는 공공후원이 더욱 확대될 필요가 있다.

현실적으로 미술후원을 위한 공적 자금을 넉넉하게 확보하는 것은 대단히 어려운 일이다. 때문에 많은 국가에서 재단에 후원하는 기업에게 세제혜택을 제공하는 등 정책적 지원을 하고 있다. 특히 영미권은 미술관에 대한 기업의 후원을 적극적으로 장려했다. 일례로 뉴욕현대미술관은 회화, 조각뿐 아니라 산업디자인이나 영상 등 넓은 분야로 분과를 확장해 다양한 기업의 참여를 유도\footnote{양은희, \snm{세계화 시대의 기업후원과 현대미술관의 변화 양상}, \bnm{기초조형학연구} 18권 5호, 한국기초조형학회, 2017, 346쪽.}하는 동시에 비주류 예술계를 꾸준히 주목해 왔다. 이처럼 독립된 예술재단과 기업이 상호이익을 얻을 수 있는 파트너십을 체결하면, 기업은 재단에 후원함으로써 세제혜택과 같은 제도적 지원과 상업적 홍보 효과를 제공받고, 재단은 후원금의 독립성을 보장받음으로써 예술적 가치를 최우선으로 다양한 미술가에게 후원을 할 수 있다. 결과적으로 미술가는 예술적 자율성을 보장받을 수 있게 된다.

\section{결론}

여기까지 예술에 후원은 필요악인가라는 물음에서 출발하여 미술후원의 세 측면으로 정치성, 경제성, 공공성을 살펴보았다. 고대로부터 권력자는 정치 선전을 위해 적극적으로 미술가를 후원해 왔다. 르네상스 시기에는 부르주아 계급이 정치적 정통성을 확보하기 위해 미술후원을 했고, 부를 쌓으며 성장한 부르주아 계급은 미술의 강력한 후원자가 되었다. 상업의 발달은 미술품을 상품화하여 시장으로 끌어들였다. 현대에는 미술품 자체가 투자상품으로 취급되기 시작했고, 수집가들은 미술품을 통해 차익을 실현하기도 했다. 이러한 흐름 속에서 기업들은 홍보 효과를 위해 미술후원을 하고 있다.

이렇게 미술사에서 후원의 역사를 되짚어 본 뒤 예술에 후원이 필요악인가 다시 질문한다면, 실제로 후원자로 인해 미술가의 자율성과 예술적 순수성이 훼손되곤 했음을 부정할 수 없다. 그러나 한편으로는 후원 자체가 당대의 예술적 가치를 투영하는 역할을 수행하기도 하기에, 미술과 미술후원을 분리해서 보려는 시도는 모순적인 접근처럼 보인다. 결국 미술후원을 가능하면 없는 것이 더 나은 필요악으로 취급하기 보다는, 건전한 미술후원을 고민하는 것이 합리적일 것이다. 앞서 논의한 바를 바탕으로, 건전한 미술후원은 (1) 미술가의 예술적 지향과 자율성을 침해하지 않고, (2) 미술가가 미술활동에 집중할 수 있는 환경을 조성하며, (3) 당대의 예술적 가치를 투영하는 동시에 비주류 미술계의 가능성을 함께 모색할 수 있는 지원이 되어야 한다. 따라서 본고에서는 이러한 미술후원의 방향으로 사익조직으로부터 독립된 예술재단이 후원의 주체가 되는 공공후원의 지속과 확대를 제안한다. 제도적 혜택과 상업적 이익을 제공함으로써 기업의 재단 후원을 유도하고, 이를 통해 재단과 미술가는 독립성과 자율성을 보장받을 수 있을 것으로 기대한다.

\begin{thebibliography}{9}
  \bibitem[김상엽, 2018]{ksy2018} 김상엽, \snm{프리마 포르타의 아우구스투스 조각상과 프로파간다}, \bnm{서양고대사연구} 52호, 한국서양고대역사문화학회, 2018.
  \bibitem[양은희, 2017]{yeh2017} 양은희, \snm{세계화 시대의 기업후원과 현대미술관의 변화 양상}, \bnm{기초조형학연구} 18권 5호, 한국기초조형학회, 2017.
  \bibitem[윤난지, 1994]{ynj1994} 윤난지, \snm{20세기 미술과 후원: 미국 모더니즘 정착에 있어서 구겐하임 재단의 역할을 중심으로}, \bnm{서양미술사학회논문집} 6호, 서양미술사학회, 1994.
  \bibitem[윤자정, 2012]{yjj2012} 윤자정, \snm{서구 미술시장의 양상: 역사적 고찰}, \bnm{현대미술학논문집} 16권 2호, 현대미술학회, 2012.
  \bibitem[이은기, 1994]{lek1994} 이은기, \snm{메디치家의 미술후원과 정치적인 목적}, \bnm{서양미술사학회논문집} 6집, 서양미술사학회, 1994.
  \bibitem[이재희, 2003]{ljh2003} 이재희, \snm{17세기 네덜란드 미술시장}, \bnm{사회경제평론} 21권, 한국사회경제학회, 2003.
  \bibitem[조명계, 2011]{jmk2011} 조명계, \snm{네덜란드 미술시장 생성과정 연구}, \bnm{문화예술경영학연구} 4권 2호, 한국문화예술경영학회, 2011.
\end{thebibliography}
