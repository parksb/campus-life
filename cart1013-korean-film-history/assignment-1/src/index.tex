\maketitle

\section*{<미몽>(1936): 일제강점기 여성담론}

% 대략적인 줄거리, 신여성 언급.
양주남 감독의 <미몽>(1936)은 필름이 남아있는 가장 오래된 한국어 발성영화다. <미몽>에서 부인 애순은 가사를 소홀히 하고 사치스러운 생활을 즐긴다. 남편은 그런 애순을 집에서 내쫓고 딸 정희와 함께 지낸다. 결국 집을 나온 애순은 옷가게에서 만난 창건과 함께 호텔에서 생활하다가, 그가 가난한 범죄자라는 사실을 알고 그를 떠난다. 이후 애순은 공연에서 본 무용가를 따라가기 위해 택시에 타 과속을 요구하는데, 빠르게 달리던 택시는 길을 건너던 정희를 치고 만다. 정희는 병원에서 치료를 받고 깨어나지만, 애순은 죄책감으로 자살을 선택한다.

% 나의 해석, 서론
나는 <미몽>을 가부장적 질서에서 벗어난 여성이 파멸적 최후를 맞는 권선징악 서사로 해석한다. <미몽>이 제작, 개봉된 1930년대는 `신여성'으로 대표되는 여성담론이 활발히 논의된 시기였다. 근대 교육을 받은 신여성은 동경의 대상인 동시에 부정한 대상으로 여겨졌으며, 남성 지식인으로부터 원색적인 비난을 받기도 했다. 남성들은 신여성을 남성 권력에 대한 도전으로 받아들였던 것이다. 따라서 나는 <미몽>이 당대 여성들을 향한 계몽적, 처벌적 시각을 투영한다고 생각한다. 이러한 해석을 뒷받침하기 위해 <미몽>이 제작된 당시의 사회문화적 배경을 살피는 과정으로 당대의 여성담론을 살펴본 뒤, 신여성을 향한 남성들의 적대적 시각을 다루며 <미몽>이 어떻게 신여성을 처벌하고 있는지 분석하고자 한다.

% 신여성에 대한 배경 설명.
애순은 신여성 내지는 모던걸을 상징한다고 볼 수 있다. 신여성은 동아시아에서 근대 교육을 받고 의식화된 여성을 지칭하는 용어로 쓰이곤 했다. 조선에서는 1919년 3$\cdot$1 운동을 전후로 민족해방 및 반봉건 의식이 여성해방 의식으로 이어지면서 신여성으로 일컬어지는 여성상이 사회 전면에 등장했다\footnote{김경일, \snm{1920-30년대 한국의 신여성과 사회주의}, \bnm{한국문화} 제36호, 규장각한국학연구소, 2005, 254쪽}. 특히 1920-30년대 신여성은 서구적 가치와 라이프스타일을 추구하면서 근대 소비문화를 향유하는 여성을 가리키는 용어인 `모던걸'과도 결합하며 전통적, 봉건적 가치를 지키는 `구여성'과 대비되는 새로운 여성상으로 인식되었다. 애순은 신여성처럼 사회의식을 드러내거나, 모던걸처럼 단발머리를 하지는 않는다. 하지만 남편에게 당당히 ``나는 조롱에 든 새는 아니니까요"라며 소리치고, 백화점에서 쇼핑을 하거나 무용을 관람한다. 심지어 집을 나온 뒤 창건과 자유연애를 즐기기도 한다. 영화 전반에서 애순은 신여성의 상징을 충분히 내비치고 있으며, 당대 관객 역시 애순을 근대의 새로운 여성상, 즉, 신여성으로 이해하기에 전혀 무리가 없었을 것이다.

% 소설, 신문, 잡지 기사에서 신여성을 비판한 내용.
신여성을 향한 남성들의 시선은 곱지 않았다. 나혜석, 김명순과 같이 자유연애론을 주창한 여성 지식인들은 잡지에서 가십거리로 소비되었다. 일간지에는 모던걸의 태도와 외모를 풍자하는 만평이 실리곤 했고, `여학생'의 이미지는 물질과 성을 탐닉하는 퇴폐적인 이미지로 덮어씌워졌다\footnote{서지영, \snm{식민지 조선의 모던걸: 1920-30년대 경성 거리의 여성 산책자}, \bnm{한국여성학} 제22권 3호, 한국여성학회, 2006, 205쪽.}. 조선의 남성 지식인들은 식민지의 이등 시민이라는 계층적 위기와 동시에, 신여성으로 인한 남성 권력의 위기를 체감한 것처럼 보인다. 이러한 위기 의식은 신여성을 향한 적대적 시각으로 이어져 소설에 반영되기도 했다. 염상섭은 소설 <제야>(1922)에서 신여성인 화자가 자신의 사상과 행동을 반성하는 모습을 연출했고, 김동인은 소설 <무능자의 안해>(1930)에서 신여성 등장인물을 그저 `선구자를 흉내 내는 여성'으로 의미화했다\footnote{박수빈, \snm{일제하 `신여성' 담론의 정치적 의미 - 김동인 소설의 미소지니즘과 여성인물 형상화를 중심으로}, \bnm{돈암어문학} 제37집, 돈암어문학회, 2020, 90쪽.}. 당대 남성들이 묘사하는 신여성이 그랬듯, 애순 역시 가정을 돌보지 않고, 성적으로 문란하며 사치와 허영이 심한 것으로 묘사된다. 이러한 애순의 캐릭터성에는 신여성에 대한 부정적 인식이 그대로, 또는 과장되어 반영된 것으로 보인다. 하지만 애순을 죽음이라는 극단적 방식으로 처벌하기에는 정당성이 부족했을 것이다. 처벌에 설득력을 부여하기 위한 장치로서 정희의 교통사고는 다소 작위적이지만 필수적이었다. 애순은 현모양처의 의무를 다하지 않았고, 이로 인해 딸이 생명의 위협을 받았다. 결국 애순은 스스로 죽음을 택함으로써 처벌받는다. 이렇게 <미몽>은 철저히 가부장적 이데올로기를 수호한다.

% 결론, 또 다른 시각, 미몽의 한계와 현재
여기까지 <미몽>이 제작, 개봉된 시기인 1920-30년대의 여성담론과 신여성에 대한 당대 남성의 관점을 살펴봤다. 그리고 이러한 시대적 배경을 바탕으로, 나는 애순이 당대의 신여성을 상징하며, <미몽>은 애순의 파멸을 통해 여성 관객을 향한 경고와 신여성에 대한 처벌적 메시지를 전달한다고 해석했다. 이광욱은 남편에게 말대꾸하는 애순의 모습을 보며 당대 여성 관객들이 대리만족을 느꼈을 것이며, 그런 애순의 모습 자체가 유의미하다고 해석한다\footnote{이광욱, \snm{초창기 조선어 발성영화의 존재 조건과 매개변수로서의 관객: <미몽>(1936)에 나타난 시청각 이미지의 양상을 중심으로}, \bnm{스토리앤이미지텔링} 제18집, 스토리앤이미지텔링연구소, 2019, 67-68쪽.}. 나는 이러한 해석에도 가능성이 있다고 생각한다. 그럼에도 불구하고, <미몽>이 오늘날에는 `필름이 남아있는 가장 오래된 한국어 발성영화' 또는 `조선 최초의 교통영화'로만 강조되는 이유는 영화가 담아내는 관점이 시대적 한계를 적극적으로 극복하지 못한 데 있을 것이다. <미몽>의 한계는 지나간 과거의 한계가 아니다. 몇 년 전까지만 해도 `여대생'의 이미지는 `명품백을 매고 스타벅스 커피를 마시는' 모습으로 묘사되곤 했다. 여성인권에 대해 말한 여성에게 인신공격성 비난이 쏟아지는 일은 최근에도 비일비재하다. 27년째 OECD 성별임금격차 1위를 차지하고 있는 한국에서 여성인권을 언급하기만 해도 `여성우월주의자'로 낙인찍혀 비난받는 현실은, 오늘날에도 여전히 신여성에 대한 처벌이 이어지고 있음을 여실히 보여준다.

\section*{<하녀>(1960): 전후복구기 계급담론}

% 대략적인 줄거리.
김기영 감독의 <하녀>(1960)는 지금도 걸작으로 평가받는 스릴러 영화다. 방직공장의 음악 강사인 동식은 임신한 아내의 가사 부담을 덜어주기 위해 여공 경희의 소개로 하녀를 고용한다. 그런데 아내와 아이들이 집을 비운 사이 동식이 하녀와 관계를 맺어 하녀가 임신하게 된다. 이 사실을 알게 된 아내는 하녀를 설득해 유산하게 만든다. 셋째를 출산한 아내를 보고 하녀는 분노하여 동식의 아들을 살해하고, 간통을 고발하겠다며 앞으로는 동식이 자신과 동침할 것을 요구한다. 그렇게 하녀가 아내의 자리를 빼앗고 지내던 중 경희가 피아노 레슨을 받기 위해 동식의 집에 찾아오는데, 하녀는 질투심에 경희를 칼로 찌른다. 칼에 찔린 경희는 경찰에 신고하겠다며 도망치고, 동식과 하녀는 쥐약으로 자살하고 만다.

% 나의 해석, 서론.
나는 <하녀>가 전후복구기 중산층이 가진 계급추락의 불안감을 투영한다고 해석한다. <하녀>가 제작, 개봉된 1950-60년대는 한국전쟁 직후 근대적 자본주의 질서가 재건되던 시기였다. 이때 소작농과 지주 사이, 노동자와 자본가 사이에서 중산층의 빈 자리를 채운 이들은 부족한 기반 위에서 언제든 빈민으로 전락할 위험을 안고 있었다. <하녀>가 공포스러운 이유는 단순히 외부인의 침입으로 한 가정이 무너지기 때문만은 아니다. 그 기저에는 기층계급에게 자리를 빼앗길지 모른다는 중산층의 불안과 공포가 있다. <하녀>가 어떻게 두 계급의 갈등을 그려내는지 분석하기 위해 우선 <하녀>가 제작된 1950년대 중산층의 의미와 동식 가정이 가진 불안감정의 배경을 살펴본다. 이어서 영화에서 `하녀'로 지칭되는 식모와 여공 사이의 사회적 지위에 어떤 차이가 있었는지 짚으며 하녀와 경희가 가진 계급상승 욕망을 다룬다.

% 중산층의 불안. 동식, 아내의 감정.
한국에서 중산층이라는 개념은 1960년대 군사정권이 `조국근대화'의 일환으로 `중산층 육성'을 경제적 목표로 추진한 뒤에야 자리 잡았다\footnote{김성국, \snm{변혁기의 한국사회 민중의 중산층화 혹은 중산층의 민중화}, \bnm{사회비평} 제1권, 나남출판사, 1988, 71쪽.}. 따라서 <하녀>가 제작된 1950년대를 기준으로 2층 주택에 거주하며, 두 자녀가 있고, 집에 피아노가 있는 동식의 가정은 전형적인 중산층의 모습이기보다는 상류층의 모습에 가까울 수도 있다. 그럼에도 동식의 가정이 자본가 계급의 조건을 만족하지 않는 이유는 생계를 위해서는 동식이 여전히 자신의 노동력을 판매해야 하며, 심지어는 가정의 생계가 아내의 재봉틀에도 상당히 의존하고 있기 때문이다. 아내의 재봉틀은 생계를 돕는 도구이기도 하지만, 동식에게는 가장으로서의 권위를 위협하는 도구이기도 하다. 이러한 현실에서 동식은 끊임없이 흔들리면서 무능한 모습을 보인다. 동식이 현실과 이상, 욕망과 신념 사이에서 갈팡질팡하다가 결국 전자 앞에 무너져 버린 것과 달리, 아내는 중산층 정상가족이라는 이상을 향해 수단과 방법을 가리지 않는다. 아내는 교양과 품격이 있는 여성으로 보이지만, 이상을 향한 아내의 집념은 하녀에게 유산을 권유하기도, 아들의 죽음을 외면하기도, 하녀의 음식에 쥐약을 넣기도 하며, 오히려 아내의 이상이나 신념과 멀어지는 결과를 초래한다. 그렇게 아내는 동식이 죽어가는 중에도 끝없이 이상을 욕망하며 쳇바퀴 도는 다람쥐처럼 재봉틀을 돌린다.

% 계급상승 욕망. 경희와 하녀의 방식.
한국의 계급담론은 해방 이후 계급혁명을 중심으로 논의되었다. 하지만 전쟁과 우익 정권의 폭력적 탄압, 미국으로부터의 자본주의 이식을 거치며 무산계급은 투쟁의 동력을 상실했고, 담론의 범위는 계급상승만으로 수축되었다. 이러한 배경하에 1950년대에는 생계유지 혹은 계급상승의 희망을 품은 농촌 여성들이 도시의 공장에 취업하며 여성의 사회진출이 크게 늘어나는 양상이 나타난다. 오늘날 `여공'의 이미지는 열악한 환경에서 착취에 가까운 장시간 노동을 하는 1960-80년대 여성의 모습이다. 그런데 이희영에 따르면 <하녀>가 제작된 1950년대 여성의 공장노동은 상대적으로 긍정적인 평가를 받았으며, 1960-80년대의 여공과는 그 지위가 달랐다\footnote{이희영, \snm{1950년대 여성노동자와 `공장노동'의 사회적 의미 - 광주 전남방직 구술 사례를 중심으로}, \bnm{산업노동연구} 제14권 제1호, 한국산업노동학회, 2008, 189-190쪽.}. 여성이 경제적 목적을 달성하기 위한 선택지로는 흔히 공장노동이나 식모살이가 있었는데, 식모살이와 달리 공장노동은 가부장적, 봉건적 질서로부터 해방될 기회이자, 경제적 자립의 기회가 되었다. 즉, 도시 공장의 임금 노동자가 되는 것은 농촌 여성의 `신분상승'에 가까웠던 것이다. <하녀>에서도 경희와 하녀의 상황은 확실히 달라 보인다. 그래서 동식을 매개로 중산층에 편입되고자 하는 경희와 달리, 하녀는 ``약속하세요. 나도 피아노를 배워주고, 미스 조마냥 껴안아 주시겠다고요. 난 죽어도 좋으니까 미스 조에게 지긴 싫어요''라며 경희가 욕망하는 것을 욕망할 뿐이다. 하지만 동식과 관계를 가진 이후로는 하녀의 계급상승 욕망이 경희의 그것을 뛰어넘을 정도로 폭발한다. 아내의 자리를 빼앗고 중산층 가정을 차지한 하녀는 목표를 달성한 것처럼 보인다. 하녀가 계단에서 떨어진 뒤 동식을 `여보'라고 부르자, 동식이 하녀를 안고 계단을 오르는 장면은 상징적이다. 그러나 하녀의 계급상승은 실패로 귀결될 수밖에 없었다. 그것이 제도적으로 보장된 형태의 상승이 아니라, 개인적인 차원에서의 상승이었기 때문이다.

% 결론.
이상으로 1950-60년대 계급담론을 배경으로 <하녀>를 분석했다. 동식과 아내는 기반이 불안정한 중산층 가정의 부부로 언제 추락할지 모른다는 불안감을 안고 있다. 기층계급인 경희와 하녀는 동식을 이용해 중산층으로의 계급상승을 꿈꾼다. <하녀>는 두 계급 사이의 갈등을 세련된 방식으로 그려냈다. 사회안전망이 부재한 사회에서, 대규모의 자본과 생산수단을 소유한 자본가와 달리 끊임없이 노동에 참여해야 하는 중산층은 늘 불안할 수밖에 없다. 마지막 장면에서 동식이 카메라를 보며 이 모든 이야기가 교훈을 주기 위한 상상이었다는 것처럼 말하는 장면은 직시하고 싶지않은 실체적 불안을 한바탕 웃음으로 애써 무마하려는 영화 관계자들의 시도처럼 느껴져 더욱 기괴하다. 60여년 전에 제작된 영화에서 그런 감정이 느껴지는 이유는 <하녀>에 투영된 계급담론이 오늘날에도 여전히 유효하기 때문일 것이다.

\begin{thebibliography}{9}
  \bibitem[김경일, 2005]{kki2005} 김경일, \snm{1920-30년대 한국의 신여성과 사회주의}, \bnm{한국문화} 제36호, 규장각한국학연구소, 2005.
  \bibitem[김성국, 1988]{ksk1988} 김성국, \snm{변혁기의 한국사회 민중의 중산층화 혹은 중산층의 민중화}, \bnm{사회비평} 제1권, 나남출판사, 1988.
  \bibitem[박수빈, 2020]{psb2020} 박수빈, \snm{일제하 `신여성' 담론의 정치적 의미 - 김동인 소설의 미소지니즘과 여성인물 형상화를 중심으로}, \bnm{돈암어문학} 제37집, 돈암어문학회, 2020.
  \bibitem[서지영, 2006]{sjy2006} 서지영, \snm{식민지 조선의 모던걸: 1920-30년대 경성 거리의 여성 산책자}, \bnm{한국여성학} 제22권 3호, 한국여성학회, 2006.
  \bibitem[이광욱, 2019]{lgw2019} 이광욱, \snm{초창기 조선어 발성영화의 존재 조건과 매개변수로서의 관객: <미몽>(1936)에 나타난 시청각 이미지의 양상을 중심으로}, \bnm{스토리앤이미지텔링} 제18집, 스토리앤이미지텔링연구소, 2019.
  \bibitem[이희영, 2008]{lhy2008} 이희영, \snm{1950년대 여성노동자와 `공장노동'의 사회적 의미 - 광주 전남방직 구술 사례를 중심으로}, \bnm{산업노동연구} 제14권 제1호, 한국산업노동학회, 2008.
\end{thebibliography}
